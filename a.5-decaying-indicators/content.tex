% DO NOT COMPILE THIS FILE DIRECTLY!
% This is included by the other .tex files.

\begin{frame}[t,plain]
\titlepage
\end{frame}

\begin{frame}
\frametitle{Indicators - Problem Statement}
    \begin{itemize}
            \item Various users and organisations can share data via MISP, multiple parties can be involved
            \begin{itemize}
                \item \textbf{Trust}, \textbf{data quality} and \textbf{time-to-live} issues
                \item Each user/organisation has \textbf{different use-cases} and interests
            \end{itemize}
        \vspace{0.5cm}
        \item Attributes can be shared in large quantities (more than 7.3 million on \texttt{MISPPRIV})
        \begin{itemize}
            \item Partial info about their validity (sightings)
            \item Partial info about their freshness (last update)
            \item Varius conflicting interests such as operational security, attribution, source reliability evaluation... (depends on the user)
        \end{itemize}
    \end{itemize}
\end{frame}

\begin{frame}
\frametitle{Sightings - Refresher}
    Sightings add temporal context to indicators.
    A user, script or an IDS can extend the information related to indicators by reporting back to MISP that
    an indicator has been \texttt{seen}, or that an indicator can be considered as a \texttt{false-positive}
    \vspace{0.5cm}
    \begin{itemize}
        \item Sightings give more credibility/visibility to indicators
        \item This information can be used to {\bf prioritise and decay indicators}
    \end{itemize}
    \begin{center}
        \includegraphics[scale=1.00]{pics/sightings.png}
    \end{center}
\end{frame}

\begin{frame}
\frametitle{Organisations opt-in - setting a level of confidence}
    MISP is a peer-to-peer system, information passes through multiple instances.
    \begin{itemize}
        \item Producers can add context (such as tags from taxonomies, galaxies) about their asserted confidence or the reliability of the data
        \item Consumers can have different levels of trust in the producers and/or analysts themselves
        \item Users might have other contextual needs
    \end{itemize}
\end{frame}

\begin{frame}
    \frametitle{Taxonomies - Refresher (1)}
    \includegraphics[width=1.00\linewidth]{pics/taxonomies.png}
\end{frame}

\begin{frame}
    \frametitle{Taxonomies - Refresher (2)}
    \includegraphics[width=1.00\linewidth]{pics/taxonomy-admiralty-scale.png}
\end{frame}

\begin{frame}
    \frametitle{Taxonomies - Refresher (3)}
    \begin{itemize}
        \item Some taxonomies have \texttt{numerical\_value}
        \begin{itemize}
            \item[$\rightarrow$] Can be used to prioritise \textit{Attributes}
        \end{itemize}
    \end{itemize}
    \vspace{1cm}

    \begin{footnotesize}
    \begin{columns}[T] % align columns
    \begin{column}{.40\textwidth}
        \begin{tabular}{|ll|}
            \hline
            \textbf{Description} & \textbf{Value}\\
            \hline
            Completely reliable & 100\\
            Usually reliable & 75\\
            Fairly reliable & 50\\
            Not usually reliable & 25\\
            Unreliable & 0\\
            Reliability cannot be judged & 50\\
            Deliberatly deceptive & 0 \textbf{\color{red}?}\\
            \hline
        \end{tabular}
    \end{column}%
    \hfill%
    \begin{column}{.48\textwidth}
        \begin{tabular}{|ll|}
            \hline
            \textbf{Description} & \textbf{Value}\\
            \hline
            Confirmed by other sources & 100\\
            Probably true & 75\\
            Possibly true & 50\\
            Doubtful & 25\\
            Improbable & 0\\
            Truth cannot be judged & 50 \textbf{\color{red}?}\\
            \hline
        \end{tabular}
    \end{column}%
    \end{columns}
    \end{footnotesize}
\end{frame}

\begin{frame}
    \frametitle{Scoring Indicators: Our solution}
    $$ \texttt{score}(\texttt{\tiny Attribute}) = \texttt{base\_score}(\texttt{\tiny Attribute}) \;\;\bullet\;\; \texttt{decay}(\texttt{\tiny Model}) $$
    Where,\vspace{0.5cm}
    \begin{itemize}
        \item \texttt{score} $ \in [0, +\infty $
        \item \texttt{base\_score} $ \in [0, 100] $
        \item \texttt{decay} is a function defined by model's parameters controlling decay speed
    \end{itemize}
    
\end{frame}

\begin{frame}
    \frametitle{Scoring Indicators: \texttt{base\_score} (1)}
        When scoring indicators\footnote{Paper available: \url{https://arxiv.org/pdf/1803.11052}}, multiple parameters\footnote{at a variable extent as required} can be taken into account. The {\bf base score} is calculated with the following in mind:
    \begin{itemize}
        \item {\color{purple}Data reliability, credibility, analyst skills, custom prioritisation tags (economical-impact), etc.}
        \item {\color{orange}Trust in the source}
    \end{itemize}
    \vspace{0.5cm}
    $$\texttt{base\_score} = \omega_{tg} \cdot {\color{purple}tags} + \omega_{sc} \cdot {\color{orange}source\_confidence}$$
\end{frame}

\begin{frame}
    \frametitle{Scoring Indicators: \texttt{base\_score} (2)}
    \includegraphics[width=1.0\linewidth]{pics/bs-computation-steps.png}
\end{frame}

\begin{frame}
    \frametitle{Scoring Indicators: decay speed (1)}
    The \texttt{score} is calculated using:
   \begin{itemize}
       \item The \texttt{lifetime} of the indicator (e.g. IP address vs hash value of a file)
       \begin{itemize}
           \item The lifespan of the indicator (short for an IP - long for an hash)
       \end{itemize}
       \item The \texttt{decay rate}, or speed at which an attribute loses value over time
   \end{itemize}
\end{frame}

\begin{frame}
    \frametitle{Scoring Indicators: putting it all toghether}
    $\rightarrow$ \texttt{decayin rate} is re-initialized upon sighting addition, or said differently, the \texttt{score} is reset to its base score as new \texttt{sightings} are received.
    $$score = base\_score \cdot \left( 1 - \left( \frac{t}{\tau_a} \right)^{\frac{1}{\delta_a}} \right) $$
\end{frame}

\begin{frame}
\frametitle{Implementation in MISP: Playing with Models}
    \begin{itemize}
        \item \textbf{Automatic scoring} based on default values
        \item \textbf{User-friendly UI} to manually set lifetime parameters
        \item \textbf{Simulation} tool
        \item Interaction through the \textbf{API}
        \item Opportunity to create your \textbf{own} formula or algorythm
    \end{itemize}
\end{frame}

\begin{frame}
    \frametitle{Implementation in MISP: Model Types}
    Multiple model types are available
    \begin{itemize}
        \item Default models: Models created and shared by the community. Available from \texttt{misp-decaying-models} repository\footnote{\url{https://github.com/MISP/misp-decaying-models.git}}.
        \begin{itemize}
            \item $\rightarrow$ Not editable
        \end{itemize}
        \item Organisation models: Models created by a user belonging to an organisation
        \begin{itemize}
            \item These models can be hidden or shared to other organisation 
            \item $\rightarrow$ Editable
        \end{itemize}
    \end{itemize}
\end{frame}

\begin{frame}
    \frametitle{Implementation in MISP: Index}
    \includegraphics[width=1.00\linewidth]{pics/decaying-index.png}
\end{frame}

\begin{frame}
    \frametitle{Implementation in MISP: Fine tuning tool}
    \includegraphics[width=1.00\linewidth]{pics/decaying-tool.png}
\end{frame}

\begin{frame}
    \frametitle{Implementation in MISP: \texttt{base\_score} tool}
    \includegraphics[width=1.00\linewidth]{pics/decaying-basescore.png}
\end{frame}

\begin{frame}
    \frametitle{Implementation in MISP: simulation tool}
    \includegraphics[width=1.00\linewidth]{pics/decaying-simulation.png}
\end{frame}

\begin{frame}
    \frametitle{Implementation in MISP: \texttt{Event/view}}
    \includegraphics[width=1.00\linewidth]{pics/decaying-event.png}
\end{frame}

\begin{frame}[fragile]
    \frametitle{Implementation in MISP: API (1)}
    \texttt{/attributes/restSearch}
    \begin{lstlisting}
{
    "includeDecayScore": 1,
    "includeFullModel": 0,
    "excludeDecayed": 0,
    "decayingModel": [85],
    "modelOverrides": {
        "threshold": 30
    }
    "score": 30,
}
    \end{lstlisting}
\end{frame}

\begin{frame}[fragile]
    \frametitle{Implementation in MISP: API (2)}
    \texttt{/attributes/restSearch}
    \begin{lstlisting}
"Attribute": [
  {
    "category": "Network activity",
    "type": "ip-src",
    "to_ids": true,
    "timestamp": "1565703507",
    [...]
    "value": "8.8.8.8",
    "decay_score": [
      {
        "score": 54.475223849544456,
        "decayed": false,
        "DecayingModel": {
          "id": "85",
          "name": "NIDS Simple Decaying Model"
        }
      }
    ],
[...]
    \end{lstlisting}
\end{frame}

\begin{frame}
    \frametitle{Creating a new decay algorithm (1)}
    The current architecture allows users to create their \textbf{own} formulae.

    \begin{itemize}
        \item Create a new file \texttt{{\$}filename} in \texttt{app/Model/DecayingModelsFormulas/}
        \item Extend the Base class as defined in \texttt{DecayingModelBase}
        \item Implement the two mandatory functions \texttt{computeScore} and \texttt{isDecayed} using your own formula/algorithm
        \item Create a Model and set the formula field to \texttt{{\$}filename}
    \end{itemize}
\end{frame}


\begin{frame}[fragile]
    \frametitle{Creating a new decay algorithm (2)}
    \lstset{basicstyle=\scriptsize}
    \begin{lstlisting}
<?php
include_once 'Base.php';

class Polynomial extends DecayingModelBase
{
    public const DESCRIPTION = 'The description of your new decaying algorithm';

    public function computeScore($model, $attribute, $base_score, $elapsed_time)
    {
       // algorithm returning a numerical score
    }

    public function isDecayed($model, $attribute, $score)
    {
        // algorithm returning a boolean stating
        // if the attribute is expired or not
    }
}
?>
    \end{lstlisting}
\end{frame}
