% DO NOT COMPILE THIS FILE DIRECTLY!
% This is included by the other .tex files.

\begin{frame}[t,plain]
\titlepage
\end{frame}

\begin{frame}
\frametitle{Objectives}
\begin{itemize}
        \item Learning how to use MISP to support common OSINT gathering use-cases as often used by SOC, CSIRTs and CERTs
        \item By using a list of practical exercise\footnote{\url{https://gist.github.com/adulau/8c1de48060e259799d3397b83b0eec4f}}
        \item The exercises are {\bf practical recent cases to model and structure intelligence} using the MISP standard
        \item Improving the data models available in MISP by exchanging live improvements and ideas
        \item Being able to share the results to the community at the end of this session
\end{itemize}
\end{frame}

\begin{frame}
\frametitle{(Threat) Intelligence}
\begin{itemize}
        \item {\bf Cyber threat intelligence (CTI) is a vast concept} which includes different fields such as intelligence as defined in the military community or in the financial sector or the intelligence community
        \item {\bf MISP project doesn't want to lock an organisation or an user into a specific model}. Each model is useful depending of the objectives from an organisation
        \item A set of pre-defined knowledge base or data-models are available and organisation can select (or create) what they need
        \item During this session, an overview of the most used taxonomies, galaxies and objects will be described
\end{itemize}
\end{frame}

\begin{frame}
        \frametitle{Overall process of collecting and analysing OSINT}
        \includegraphics[scale=0.17]{OSINT_MISP_almostcomplete.png}
\end{frame}

\begin{frame}
\frametitle{Meta information and contextualisation 1/2}
\begin{itemize}
\item Quality of indicators/attributes are important but {\bf tagging and classification are also critical to ensure actionable information}
        \item Tagging intelligence is done by using tags in MISP which are often originating from MISP taxonomy libraries
        \item The scope can be classification ({\it tlp, PAP}), type ({\it osint, type, veris}), state ({\it workflow}), collaboration ({\it collaborative-intelligence}) and many other fields
        \item MISP taxonomies documentation is available\footnote{\url{https://www.misp-project.org/taxonomies.html}}
        \item {\bf Review existing practices of tagging in your sharing community, reuse practices and improve context}
\end{itemize}
\end{frame}

\begin{frame}
\frametitle{Meta information and contextualisation 2/2}
\begin{itemize}
        \item {\bf When information cannot be expressed in triple tags format} ({\it namespace:predicate=value}), MISP provides the galaxies
        \item Galaxies contain a huge set of common libraries\footnote{\url{https://www.misp-project.org/galaxy.html}} such as threat actors, malicious tools, RAT, Ransomware, target information and many more
        \item When tagging or adding a galaxy cluster, don't forget that tagging at event level is for the whole event (including attributes and objects). While tagging at attribute level, it's often a more specific context
\end{itemize}
\end{frame}

\begin{frame}
\frametitle{Adding attributes/objects to an event}
\begin{itemize}
        \item If the information is a {\bf single atomic element}, using a single attribute is preferred
                \begin{itemize}
                        \item Choosing an attribute type is critical as this defines the automation/export rule (e.g. url versus link or ip-src/ip-dst?)
                        \item Enabling the IDS (automation) flag is also important. When you are in doubt, don't set the IDS flag
                \end{itemize}
        \item If the information is {\bf composite} (ip/port, filename/hash, bank account/BIC), using a object is strongly recommended
\end{itemize}
\end{frame}

\begin{frame}
       \frametitle{How to select the right object?}

        There are more than 150 MISP objects\footnote{\url{https://www.misp-project.org/objects.html}} templates.\\
      As an example, at CIRCL, we regularly use the following object templates {\it file}, {\it microblog}, {\it domain-ip}, {\it ip-port}, {\it coin-address}, {\it virustotal-report}, {\it paste}, {\it person}, {\it ail-leak}, {\it pe}, {\it pe-section}, {\it registry-key}.\\
\end{frame}

\begin{frame}
\frametitle{microblog object}
\begin{columns}[totalwidth=\textwidth]
        \column{0.49\textwidth}\underline{Use case}\\
A serie of OSINT tweets from a security researcher.
To structure the thread, the information
and keep an history.\\
        \includegraphics[scale=0.15]{emotet.png}
        \column{0.49\textwidth}\underline{Object to use}\\
        The microblog object can be used for Tweet or any microblog post (e.g. Facebook). Then object can be linked using {\it followed-by} to describe a serie of post.\\
        \includegraphics[scale=0.15]{microblog.png}
\end{columns}
\end{frame}


\begin{frame}
\frametitle{file object}
\begin{columns}[totalwidth=\textwidth]
        \column{0.49\textwidth}\underline{Use case}\\
        \begin{itemize}
                \item A file sample was received by email or extracted from VirusTotal.
                \item A list of file hashes were included in a report.
                \item A hash value was mentioned in a blog post.
        \end{itemize}
        \column{0.49\textwidth}\underline{Object to use}\\
        The file object can be used to describe file. It's usual to have partial meta information such as a single hash and a filename.\\
        \includegraphics[scale=0.25]{fileobject.png}
\end{columns}
\end{frame}

\begin{frame}
        \frametitle{References}
        \begin{itemize}
        \item Graphical overview of OSINT collection using MISP \url{https://github.com/adulau/misp-osint-collection}
        \item MISP objects documentation \url{https://www.misp-project.org/objects.html}
        \item MISP taxonomies documentation \url{https://www.misp-project.org/taxonomies.html}
        \item MISP galaxy documentation \url{https://www.misp-project.org/galaxy.html}
        \end{itemize}
\end{frame}

