\documentclass{beamer}
\usetheme[numbering=progressbar]{focus}
\definecolor{main}{RGB}{47, 161, 219}
\definecolor{textcolor}{RGB}{128, 128, 128}
\definecolor{background}{RGB}{240, 247, 255}
\definecolor{mybeige}{HTML}{eeeeee}
\definecolor{mymauve}{rgb}{0.58,0,0.82}
\definecolor{myblack}{rgb}{0,0,0}

\usepackage[utf8]{inputenc}
\usepackage{tikz}
\usetikzlibrary{shapes,snakes,automata,positioning}
\usepackage{listings}
\usepackage{adjustbox}
%\usepackage[T1]{fontenc}
%\usepackage[scaled]{beramono}
\author{\small{Iklódy András}}
\title{Cyber-threat információ-megosztó közösségek építése}
\date{ITBN 2020}
\subtitle{8 év tanulságai}
\titlegraphic{\includegraphics[scale=0.85]{pics/misp.pdf}}

\lstdefinestyle{code}{ %
  backgroundcolor=\color{mybeige},   % choose the background color; you must add \usepackage{color} or \usepackage{xcolor}; should come as last argument
  basicstyle=\footnotesize\ttfamily,        % the size of the fonts that are used for the code
  breakatwhitespace=false,         % sets if automatic breaks should only happen at whitespace
  breaklines=true,                 % sets automatic line breaking
  captionpos=b,                    % sets the caption-position to bottom
  commentstyle=\color{mygreen},    % comment style
  deletekeywords={...},            % if you want to delete keywords from the given language
  escapeinside={\%*}{*)},          % if you want to add LaTeX within your code
  extendedchars=true,              % lets you use non-ASCII characters; for 8-bits encodings only, does not work with UTF-8
  frame=single,	                   % adds a frame around the code
  keepspaces=true,                 % keeps spaces in text, useful for keeping indentation of code (possibly needs columns=flexible)
  keywordstyle=\color{blue},       % keyword style
  language=Python,                 % the language of the code
  morekeywords={*,...},           % if you want to add more keywords to the set
  numbers=left,                    % where to put the line-numbers; possible values are (none, left, right)
  numbersep=5pt,                   % how far the line-numbers are from the code
  numberstyle=\tiny\color{myblack}, % the style that is used for the line-numbers
  rulecolor=\color{black},         % if not set, the frame-color may be changed on line-breaks within not-black text (e.g. comments (green here))
  showspaces=false,                % show spaces everywhere adding particular underscores; it overrides 'showstringspaces'
  showstringspaces=false,          % underline spaces within strings only
  showtabs=false,                  % show tabs within strings adding particular underscores
  stepnumber=1,                    % the step between two line-numbers. If it's 1, each line will be numbered
  stringstyle=\color{mymauve},     % string literal style
  tabsize=2,	                   % sets default tabsize to 2 spaces
  title=\lstname                   % show the filename of files included with \lstinputlisting; also try caption instead of title
}
\lstset{style=code}

\begin{document}
% DO NOT COMPILE THIS FILE DIRECTLY!
% This is included by the other .tex files.

\begin{frame}[t,plain]
\titlepage
\end{frame}

\begin{frame}
    \frametitle{Who am I}
    \begin{minipage}{0.6\textwidth}
        \begin{itemize}
            \item \faGithub : chrisr3d \\
            \item \faMastodon : @chrisr3d@infosec.exchange
            \item \faTwitter : chrisred\_68
            \item []
            \item Interoperability Wizard @ CIRCL
            \item MISP core development team
            \item STIX WG co-chair
            \item []
            \item \faCat \vspace{1em} \& \faCamera \vspace{1em} enthusiast
        \end{itemize}
    \end{minipage}%
    \begin{minipage}{0.4\textwidth}
        \includegraphics[scale=0.1]{images/profile_picture.jpg}
    \end{minipage}
\end{frame}

\begin{frame}
    \frametitle{Summary}
    \begin{itemize}
        \item From an ocean of unknown errors...\linebreak $\Rightarrow$ the difficulty to parse STIX content
        \item ... To a more \& more accurate support\linebreak $\Rightarrow$ \emph{misp-stix} - The Holy Grail for MISP \& STIX
        \item ... And even further\linebreak $\Rightarrow$ Evolution \& improvement perspectives
        \item The magic word: \emph{interoperability}
        \item Demo (?)
    \end{itemize}
\end{frame}

\begin{frame}
    \frametitle{STIX - Quick recap}
    \begin{minipage}{0.5\textwidth}
        \centering
        \includegraphics[scale=0.5]{images/LOGO_STIX.pdf}
    \end{minipage}%
    \begin{minipage}{0.5\textwidth}
        \centering
        \includegraphics[scale=0.45]{images/LOGO_TAXII.pdf}
    \end{minipage}
    \vspace{1em}
    \begin{itemize}
        \item \textbf{S}tructured \textbf{T}hreat \textbf{I}ntelligence E\textbf{x}pression
        \begin{itemize}
            \item Focused on \textbf{Threat Intelligence} exchange
            \item 2 major versions with different formats
            \begin{itemize}
                \item 1.x - \emph{mostly} XML
                \item 2.x - JSON
            \end{itemize}
        \end{itemize}
        \item \textbf{T}rusted \textbf{A}utomated E\textbf{x}change of \textbf{I}ntelligence \textbf{I}nformation
        \begin{itemize}
            \item Exchange Protocol
            \item Specifically designed to support the exchange of \textbf{CTI} represented in STIX
        \end{itemize}
    \end{itemize}
\end{frame}

\begin{frame}
    \frametitle{\emph{misp-stix} - The Holy Grail for MISP \& STIX interactions}
    \centering
    \includegraphics[scale=0.3]{images/solution.png}\footnote{Python 3.8 required}
    \setcounter{footnote}{0}
\end{frame}

\end{document}

