% DO NOT COMPILE THIS FILE DIRECTLY!
% This is included by the other .tex files.

\lstdefinelanguage{json}{
    basicstyle=\ttfamily\footnotesize,
    numbers=left,
    numberstyle=\ttfamily\footnotesize,
    stepnumber=1,
    numbersep=8pt,
    showstringspaces=false,
    breaklines=true,
    frame=lines,
    backgroundcolor=\color{background},
    literate=
     *{0}{{{\color{numb}0}}}{1}
      {1}{{{\color{numb}1}}}{1}
      {2}{{{\color{numb}2}}}{1}
      {3}{{{\color{numb}3}}}{1}
      {4}{{{\color{numb}4}}}{1}
      {5}{{{\color{numb}5}}}{1}
      {6}{{{\color{numb}6}}}{1}
      {7}{{{\color{numb}7}}}{1}
      {8}{{{\color{numb}8}}}{1}
      {9}{{{\color{numb}9}}}{1}
      {:}{{{\color{punct}{:}}}}{1}
      {,}{{{\color{punct}{,}}}}{1}
      {\{}{{{\color{delim}{\{}}}}{1}
      {\}}{{{\color{delim}{\}}}}}{1}
      {[}{{{\color{delim}{[}}}}{1}
      {]}{{{\color{delim}{]}}}}{1},
}

\begin{frame}[t,plain]
\titlepage
\end{frame}

\begin{frame}
    \frametitle{Context}
    \begin{itemize}
        \item You receive emails with IoC's inside
        \item How to create an event out of it?
        \item Create event manually and copy paste
        \item $\to$ This works once or twice
        \item Forwarding the email would be nice
        \item $\to$ mail\_to\_misp
    \end{itemize}
\end{frame}

\begin{frame}
    \frametitle{Features: Email handling}
    \begin{itemize}
        \item Extraction of URLs and IP addresses and port numbers
        \item Extraction of hostnames from URLs
        \item Extraction of hashes (MD5, SHA1, SHA256)
        \item DNS expansion
        \item Subject filters
        \item Refanging of URLs ('hxxp://...')
        \item ... and more
    \end{itemize}
\end{frame}

\begin{frame}
    \frametitle{Features: Support MISP features}
    \begin{itemize}
        \item Add tags automatically
        \item Ignore 'whitelisted' domains
        \item Configurable list of attributes not to enable the IDS flag
        \item DNS expansion
        \item Automatically create 'external analysis' links based on filter list (e.g. VirusTotal, malwr.com)
        \item Automatically filter out attributes that are on a server side warning list
        \item Support for value sighting
        \item ... and more
    \end{itemize}
\end{frame}

\begin{frame}
    \frametitle{Implementation}
    \begin{itemize}
        \item Legacy
        \begin{itemize}
            \item Email $\to$ Apple Mail $\to$ Mail rule $\to$ AppleScript
            \item[] $\to$ AppleScript $\to$ mail\_to\_misp $\to$ PyMISP $\to$ MISP
            \item[]
            \item Email $\to$ Thunderbird $\to$ Mail rule $\to$ filterscript $\to$
            \item[]thunderbird\_wrapper $\to$ mail\_to\_misp $\to$ PyMISP $\to$ MISP
        \end{itemize}
        \item[]
        \item Postfix and others
        \begin{itemize}
            \item Email $\to$ mail\_to\_misp
        \end{itemize}
    \end{itemize}
\end{frame}

\begin{frame}[fragile]
    \frametitle{Installation}
    \begin{itemize}
        \item mail\_to\_misp
        \begin{enumerate}
            \item \texttt{git clone git://github.com/MISP/mail\_to\_misp.git}
            \item Install dependencies - See Github site
        \end{enumerate}
        \item[]
        \item MTA (Postfix or alike)
        \begin{enumerate}
            \item Setup a new email address in the aliases file (e.g. /etc/aliases)
            \item[] \texttt{misp\_handler: "|/path/to/mail\_to\_misp.py -"}
            \item Rebuild the DB
            \item[] \texttt{sudo newaliases}
            \item Configure mail\_to\_misp\_config.py
\begin{lstlisting}[basicstyle=\tiny]
misp_url = 'http://127.0.0.1/'
misp_key = 's5jPWCIud36Z8XHgsiCVI7SaL1XsMTyfEsN45tTe'
misp_verifycert = True
body_config_prefix = 'm2m'
...
...
\end{lstlisting}
        \end{enumerate}
    \end{itemize}
\end{frame}

\begin{frame}[fragile]
    \frametitle{Exercise: mail\_2\_misp.py}
    \begin{itemize}
        \item Bonus: \texttt{https://github.com/MISP/mail\_to\_misp\_test}
\begin{lstlisting}[basicstyle=\tiny]
./mail_to_misp.py -r mail_to_misp_test/simple_forward.eml
\end{lstlisting}
        \item Bonus: Fake-SMTPD spamtrap
\begin{lstlisting}[basicstyle=\tiny]
./fake_smtp.py

telnet 127.0.0.1 2526
    Trying 127.0.0.1...
    Connected to 127.0.0.1.
    Escape character is '^]'.
    220 misp Python SMTP 1.1
    helo misp
    250 misp
    mail from: mikel
    250 OK
    rcpt to: m2m
    250 OK
    data
    354 End data with <CR><LF>.<CR><LF>

\end{lstlisting}
    \end{itemize}
\end{frame}



